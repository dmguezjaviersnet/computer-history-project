\chapter{El microprocesador}\label{chapter:microprocesador}


El microprocesador es un procesador de computadora, donde la lógica del procesamiento de datos y el control
están incluidos en un solo circuito integrado, o en un pequeño número de circuitos integrados.
El microprocesador contiene los circuitos aritméticos, lógicos y de control necesarios para realizar 
las funciones de la unidad central de procesamiento  de una computadora. El circuito integrado es capaz de interpretar y 
ejecutar instrucciones de programa y realizar operaciones aritméticas. 
El microprocesador es un circuito integrado digital multipropósito, controlado por reloj y basado en registros que 
acepta datos binarios como entrada, los procesa de acuerdo con las instrucciones almacenadas en su memoria y proporciona 
resultados (también en forma binaria) como salida. Un microprocesador hipotético minimamente funcional podría incluir
solo una \textbf{ALU}(\emph{aritmetic logic unit }) y una sección lógica de control. La \textbf{ALU} realiza sumas, restas 
y operaciones como \textbf{AND} y \textbf{OR}. Cada operación de la \textbf{ALU} establece uno o más \emph{flags} en un registro 
de estados, que indican los resultados de la última operación(por ejemplo, si el resultado es cero, si es negativo, si 
hay desbordamiento, etc.). La lógica de control recupera los código de instrucción desde la memoria e inicia la secuencia de
operaciones necesarias parar que la \textbf{ALU} lleve a cabo la instrucción\brackcite{wikipedia_2022_Microprocesador}.
\\ Antes de los microprocesadores las computadoras pequeñas se construían utilizando  \emph{racks} de placas de circuito con 
muchos circuitos integrados de mediana y pequeña escala, generalmente del tipo \textbf{TTL}\emph{(Transistor-Transistor Logic)} 
\brackcite{wikipedia_2022_ttl}. Los microprocesadores combinaron esto en uno o unos pocos circuitos integrados a gran escala.\\
El incremento continuo de las capacidades de los microprocesadores desde  que 
se empezaron a fabricar ha dejado otras formas de computadoras casi complemtamente
obsoletas, con uno o más microprocesadores usados en todo desde pequeños sistemas 
embebidos  y dispostivos portátiles hasta los enormes \emph{mainframes} y las supercomputadoras

\section{Surgimiento del microprocesadaor}
Las invenciones a veces ocurren cuando las personas se enfrentan a un problema y luchan por resolverlo. En otras ocasiones, suceden 
cuando las personas adoptan una meta visionaria. La historia de cómo Ted Hoff y su equipo de Intel inventaron el microprocesador es 
un caso de ambos \brackcite{isaacson_2014}. Hoff, que había sido un joven profesor en Stanford, se convirtió en el duodécimo empleado 
de Intel, donde fue asignado para trabajar en el diseño de chips. Se dio cuenta de que era un desperdicio y poco elegante diseñar 
muchos tipos de microchips que cada uno tuviera una función diferente, lo que Intel estaba haciendo. Llegaría una empresa y le pediría 
que construyera un microchip diseñado para realizar una tarea específica. Hoff imaginó, al igual que Noyce y otros, un enfoque 
alternativo: crear un chip de propósito general que pudiera ser instruido o programado para realizar una variedad de aplicaciones 
diferentes según se desee. En otras palabras, una computadora de propósito general en un chip.


\subsection{Intel 4004}


