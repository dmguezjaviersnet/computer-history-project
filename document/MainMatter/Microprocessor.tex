\chapter{El microprocesador}\label{chapter:microprocesador}

El \textbf{microprocesador} es un procesador de computadora, donde la lógica del procesamiento de datos y el control
están incluidos en un solo circuito integrado, o en un pequeño número de circuitos integrados. El \textbf{microprocesador}
contiene los circuitos aritméticos, lógicos y de control necesarios para realizar las funciones de la unidad central de
procesamiento  de una computadora. El circuito integrado es capaz de interpretar y ejecutar instrucciones de programa y
realizar operaciones aritméticas. El \textbf{microprocesador} es un circuito integrado digital multipropósito, controlado
por reloj y basado en registros que acepta datos binarios como entrada, los procesa de acuerdo con las instrucciones
almacenadas en su memoria y proporciona resultados (también en forma binaria) como salida. Un \textbf{microprocesador}
hipotético mínimamente funcional podría incluir solo una \textbf{ALU}(\emph{Arithmetic Logic Unit}) y una sección lógica
de control. La \textbf{ALU} realiza sumas, restas y operaciones como \textbf{AND} y \textbf{OR}. Cada operación de la
\textbf{ALU} establece uno o más \emph{flags} en un registro de estados, que indican los resultados de la última operación
(por ejemplo, si el resultado es cero, si es negativo, si hay desbordamiento, etc.). La lógica de control recupera los
código de instrucción desde la memoria e inicia la secuencia de operaciones necesarias parar que la \textbf{ALU} lleve a
cabo la instrucción \brackcite{wikipedia_2022_Microprocesador}.\\ Antes de los microprocesadores las computadoras pequeñas
se construían utilizando  \emph{racks} de placas de circuito con muchos circuitos integrados de mediana y pequeña escala,
generalmente del tipo \textbf{TTL}\emph{(Transistor-Transistor Logic)} \brackcite{wikipedia_2022_ttl}. Los microprocesadores
combinaron esto en uno o unos pocos circuitos integrados a gran escala.\\ El incremento continuo de las capacidades de los
microprocesadores desde  que se empezaron a fabricar ha dejado otras formas de computadoras casi complemtamente obsoletas,
con uno o más microprocesadores usados en todo desde pequeños sistemas embebidos y dispostivos portátiles hasta los enormes
\emph{mainframes} y las supercomputadoras

\section{Surgimiento del microprocesadaor}
Las invenciones a veces ocurren cuando las personas se enfrentan a un problema y luchan por resolverlo. En otras ocasiones, suceden 
cuando las personas adoptan una meta visionaria. La historia de cómo \emph{Ted Hoff} y su equipo de \emph{Intel} inventaron el \textbf
{microprocesador} es un caso de ambos \brackcite{isaacson_2014}. \emph{Hoff}, que había sido un joven profesor en \emph{Stanford}, se
convirtió en el duodécimo empleado de \emph{Intel}, donde fue asignado para trabajar en el diseño de chips. \emph{Hoff} se dió cuenta de que era un
desperdicio y poco elegante diseñar muchos tipos de microchips donde cada uno tuviera una función diferente,  que era lo que \textbf
{Intel} estaba haciendo hasta ahora. Llegaría una empresa y le pediría que construyera un \textbf{microchip} diseñado para realizar una tarea
específica. \emph{Hoff} imaginó, al igual que \emph{Noyce} y otros, un enfoque alternativo: crear un \textbf{chip} de propósito general que pudiera
ser instruido o programado para realizar una variedad de aplicaciones diferentes según se desee. En otras palabras, una computadora de propósito
general en un \textbf{chip}. Esta visión coincidió con un problema que se planteó a \emph{Intel} en el verano de 1969. Una empresa japonesa llamada
\emph {Busicom} estaba planeando una nueva y poderosa calculadora de escritorio : la \textbf{Busicom 141-PF}, y había elaborado especificaciones para
doce microchips de propósito especial para diseñar 12 chips personalizados para su nueva calculadora de impresión (diferentes para manejar pantalla,
cálculos, memoria, etc.) que quería que \emph{Intel} construyera. \emph{Intel} estuvo de acuerdo y se fijó un precio. \emph{Noyce} le pidió a \emph
{Hoff} que supervisara el proyecto. La cantidad de chips y su complejidad fue mucho mayor de lo que esperaba, por lo que no había forma de que \emph
{Intel} pudidiera construirlos al precio acordado. \emph{Intel} se propuso diseñar un solo \textbf{chip} lógico que pudiera realizar casi todas las
tareas que quería \emph{Busicom}. Para empeorar las cosas,  la creciente popularidad de la calculadora de bolsillo de \emph{Jack Kilby} estaba
obligando a \emph{Busicom} a reducir aún más su precio. \emph{Noyce} le dijo a \emph{Intel} que lo intentara y tuvo que convencer a Grove del
proyecto antes de venderle la idea a \emph{Busicom}. En septiembre de 1969, \emph{Intel} y su colega \emph{Stan Mazor} habían esbozado
la arquitectura de un \textbf{chip} lógico de uso general que podía seguir instrucciones de programación. Sería capaz de hacer el trabajo de
nueve de los doce chips que había solicitado \emph{Busicom}. Se diseñaron un conjunto cuatro chips conocido como \textbf{MCS-4}. Incluía un
\textbf{chip} de unidad de procesamiento central (\textbf{CPU}), el \textbf{4004}, así como un \textbf{chip} de memoria de solo lectura
(\textbf{ROM}) compatible para los programas de aplicaciones personalizadas, un \textbf{chip} de memoria de acceso aleatorio (\textbf{RAM})
para procesar datos, y un \textbf{chip} de registro para el puerto de entrada/salida (E/S). Cuando llegó el momento de renegociar el precio,
\emph{Intel} hizo una recomendación fundamental a \emph{Noyce}, que ayudó a crear un gran mercado para los chips de uso general y aseguró que
\emph{Intel} seguiría siendo un impulsor de la era digital. A cambio de darle a \emph{Busicom} un buen precio, \emph{Noyce} insistió en que
\emph{Intel} retuviera los derechos del nuevo \textbf{chip} y se le permitiera licenciarlo a otras compañías para fines distintos a la
fabricación de una calculadora. Debido a que era esencialmente un procesador de computadora en un \textbf{chip}, el nuevo dispositivo se
denominó \textbf{microprocesador}. En noviembre de 1971 \emph{Intel} dio a conocer el producto, el \textbf{Intel 4004}, al público.
Este tenía un precio de salida de \$200 \brackcite{intel_4004, wikipedia_2022_Microprocesador,campbell-kelly_garcia-swartz_2015,isaacson_2014}.

\begin{figure}[htb]
	\centering
	\includegraphics[scale = 0.25]{Graphics/faggin_hoff_mazor_-4004.jpg}
	\caption{Desde la izquierda, \emph{Federico Faggin}, \emph{Ted Hoff} y \emph{Stanley Mazor} con procesadores \textbf{Intel 4004}}
	\label{fig:10}
\end{figure}

\subsection{Intel 4004}
Este revolucionario \textbf{microprocesador}, del tamaño de la uña del dedo meñique, entregaba la misma potencia de cómputo que la primera
computadora electrónica construida en 1946, que llenó una habitación entera. El primer \textbf{microprocesador} \textbf{Intel 4004} se
fabricó en obleas de dos pulgadas, en comparación con las obleas de 12 pulgadas que se utilizan habitualmente en los productos actuales.
El \textbf {microprocesador} \textbf{Intel 4004} es único en el sentido de que es uno de los diseños de \textbf{microprocesador} más
pequeños que alguna vez entró en producción comercial. El \textbf{4004}  era un \textbf{microprocesador} de \textbf{4-bits}, alcanzaba una máxima
velocida de reloj 740 kHz. El ancho de línea del circuito del \textbf{microprocesador} \textbf{Intel 4004} era de 10 micrones o 10 000
nanómetros. En 1971 el \textbf{Intel 4004} contenía 2300 transistores Para el 2010, un procesador \textbf{Intel Core} con matriz de procesamiento
de 32 nm y tecnología de \textbf{silicio} de puerta metálica de alta k de segunda generación contenía  560 millones de transistores. En comparación,
un cabello humano promedio tiene 100,000 nanómetros de ancho \brackcite{intel_4004,wikipedia_2022_intel_4004}.      

\begin{figure}[htb]
	\centering
	\includegraphics[scale = 0.15]{Graphics/Intel_C4004.jpg}
	\caption{Primer \textbf{microprocesador}: \textbf{Intel 4004}}
	\label{fig:12}
\end{figure}


\section{El arribo del primer microprocesador de 8-bits}
El \textbf{Intel 4004} fue seguido en 1972 por el \textbf{Intel 8008}, el primer \textbf{microprocesador} de 8 bits del mundo.
Sin embargo, el \textbf{8008} no fue una extensión del diseño del \textbf{4004}, sino la culminación de un proyecto de diseño separado
en \emph{Intel}, que surgió de un contrato con \emph{Computer Terminals Corporation}(\textbf{CTC}), por un \textbf{chip} para
una terminal que estaban diseñando: el \textbf{Datapoint 2200}, los aspectos fundamentales del diseño no provinieron de \emph{Intel}
sino de \textbf{CTC}. El \textbf{8008} salió con 3.500 transistores, tenía un bus de datos de 8 bits y un bus de direcciones de 14 bits. Este
\textbf{micoprocesador} fue la base del famoso kit de computadora "Mark-8" \brackcite{wikipedia_2022_intel_8008, campbell-kelly_garcia-swartz_2015}.

\begin{figure}[htb]
	\centering
	\includegraphics[scale = 0.15]{Graphics/Intel_C8008-1.jpg}
	\caption{\textbf{Microprocesador}  \textbf{Intel 8008}}
	\label{fig:13}
\end{figure}

\subsection{8080}
El Intel \textbf{8080} es el segundo \textbf{microprocesador} de 8 bits diseñado y fabricado por Intel. Apareció por primera vez en abril de
1974 y es una variante ampliada y mejorada del diseño anterior del \textbf{8008}, aunque sin compatibilidad binaria. La velocidad de reloj o
límite de frecuencia especificado inicialmente era de 2 MHz, con instrucciones comunes que usaban 4, 5, 7, 10 o 11 ciclos. Como resultado,
el procesador podía ejecutar varios cientos de miles de instrucciones por segundo. Federico Faggin lo concibió y diseñó utilizando MOS de
canal N de alto voltaje.

\begin{figure}[htb]
	\centering
	\includegraphics[scale = 0.2]{Graphics/8080_microprocessorr.jpg}
	\caption{\textbf{Microprocesador} \textbf{Intel 8080}}
	\label{fig:14}
\end{figure}

\section{El arribo de los 16 bits}
El primer \textbf{microprocesador} multichip de 16 bits fue el \textbf{IMP-16} de \emph{National Semiconductor}, presentado a principios de
1973. Una versión de 8 bits del conjunto de chips se introdujo en 1974 como \textbf{IMP-8}. Durante el mismo año, \emph{National} presentó
el primer \textbf{microprocesador} de un solo \textbf{chip} de 16 bits, el \textbf{PACE}, seguido de una versión NMOS, el \textbf{INS8900}.
El primer \textbf{microprocesador} de 16 bits de un solo chip fue el TMS 9900 de \emph{Texas Instruments}, presentado en 1976, que también
era compatible con su línea de minicomputadoras \textbf{TI-990}. \emph{Intel} produjo su primer procesador de 16 bits, el \textbf{8086},
en 1978. Era compatible con el \textbf{8080} y el \textbf{8085} (un derivado del \textbf{8080}) \brackcite{wikipedia_2022_Microprocesador,
staff_2021_microprocessor}.

\begin{figure}[htb]
	\centering
	\includegraphics[scale = 0.2]{Graphics/NSIMP-16A.jpg}
	\caption{\textbf{Microprocesador} \textbf{IMP-16}}
	\label{fig:15}
\end{figure}

\section{8086: El comienzo de x86}
El primer procesador de 16 bits de \emph{Intel} fue el \textbf{8086}, que ayudó a mejorar considerablemente el rendimiento en comparación
con los diseños anteriores. Este \textbf{microprocesador} utilizaba la misma microarquitectura que los microprocesadores de 8 bits de Intel
(\textbf{8008}, \textbf{8080} y \textbf{8085}). Esto permitió que los programas en lenguaje ensamblador escritos en 8 bits migraran sin
problemas. El \textbf{8086} no solo tenía una frecuencia más alta que la \textbf{8088}, sino que también empleaba un bus de datos externo
de 16 bits y una cola de precarga de 6 bytes más larga. También podía ejecutar tareas de 16 bits (aunque la mayoría del \emph{software} en
ese momento estaba diseñado para procesadores de 8 bits). El bus de direcciones se amplió a 20 bits, lo que permitió al \textbf{8086} acceder
a hasta 1 MB de memoria y, por lo tanto, aumentar el rendimiento. El \textbf{8086} también se convirtió en el primer procesador x86 y utilizó
la primera revisión del x86 ISA \brackcite{wikipedia_2022_x86_ISA}, en el que se han basado casi todos los procesadores creados por \emph{AMD}
o \emph{Intel} desde la introducción del \textbf{8086} \brackcite{wikipedia_2022_Microprocesador, sexton_2018_history_of_intel_cpus,  wikipedia_2022_intel_8086}.

\begin{figure}[htb]
	\centering
	\includegraphics[scale = 0.2]{Graphics/Intel_C8086.jpg}
	\caption{\textbf{Intel 8086}}
	\label{fig:16}
\end{figure}

\section{El arribo de los 32 bits}
Los diseños de 16 bits solo habían estado en el mercado brevemente cuando comenzaron a aparecer implementaciones de 32 bits.
El más significativo de los diseños de  microprocesadores de 32 bits fue el \textbf{Motorola MC68000}, introducido en 1979. 
El \textbf{68k}, como se le conocía ampliamente, tenía  registros de 32 bits en su modelo de programación pero usaba rutas internas
de datos de 16 bits, tres unidades aritméticas-lógicas de 16 bits, y un bus de datos externo de 16 bits para reducir 
el conteo de pines. Soportaba externamente solo direcciones de 24 bits (internamente funcionaba con direcciones completas de 
32 bits). Los diseños de \textbf{Apple Lisa} y \textbf{Macintosh} utilizaron el \textbf{68k}, al igual que muchos otros diseños a
mediados de la década de 1980, incluidos \textbf{Atari ST} y \textbf{Commodore Amiga}. El primer \textbf{microprocesador} de 32 bits de
un solo \textbf{chip} del mundo, con rutas de datos de 32 bits, buses de 32 bits y direcciones de 32 bits, fue el \textbf{AT\&T Bell Labs BELLMAC-32A},
con las primeras muestras en 1980 y la producción general en 1982. Después de la venta de \emph{AT\&T} en 1984, pasó a llamarse \textbf{WE 32000}
(de \emph{Western Electric}) y tuvo dos generaciones posteriores, \textbf{WE 32100} y \textbf{WE 32200}. El primer microprocesador comercial de
un solo \textbf{chip} de 32 bits disponible en el mercado fue el HP FOCUS. El primer \textbf{microprocesador} de 32 bits de \emph{Intel} fue el \textbf
{iAPX 432}, que se introdujo en 1981, pero no fue un éxito comercial. Tenía una arquitectura orientada a objetos basada en capacidades avanzadas,
pero un rendimiento deficiente en comparación con las arquitecturas contemporáneas, como la \textbf{80286} de \emph{Intel}(introducida en 1982),
que era casi cuatro veces más rápida en las pruebas de referencia típicas. Sin embargo, los resultados para el \textbf{iAPX 432} se debieron en parte a un
compilador de Ada apresurado y, por lo tanto, subóptimo.

\begin{figure}[htb]
	\centering
	\includegraphics[scale = 0.15]{Graphics/Motorola_MC68HC000FN8-0695.jpg}
	\caption{\textbf{Motorola 68K}}
	\label{fig:17}
\end{figure}
\subsection{80386: x86 se convierte a 32-bit}
El primer procesador x86 de 32 bits de Intel fue el 80386(renombrado más tarde como i386), lanzado en 1985. Una ventaja clave que tenía este procesador era su bus de 
direcciones de 32 bits que le permitía admitir hasta 4 GB de memoria del sistema. Aunque esto era mucho más de lo que nadie estaba 
usando en ese momento, las limitaciones de RAM a menudo perjudican el rendimiento de los procesadores x86 anteriores y de la competencia.
\begin{figure}[htb]
	\centering
	\includegraphics[scale = 0.1]{Graphics/Intel_i386DX.jpg}
	\caption{Intel i386}
	\label{fig:18}
\end{figure}
\newpage
\section{El arribo de los 64 bits}
Antes que se introdujeran los primeros microprocesadores de 64 bits destinados a las computadoras peronsales, desde principios de la década de
1990 se utilizaron diseños de microprocesadores de 64 bits en varios mercados. En 1991 \textbf{MIPS Computer Systems} produce el primer \textbf
{microprocesador} de 64 bits, el \textbf{R4000}, que implementa la arquitectura \textbf{MIPS III}, la tercera revisión de su arquitectura \textbf{MIPS}.
El \textbf{R4000} se utilizó en estaciones de trabajo de gráficos SGI. En 1996 la compañía nipona de videojuegos \emph{Nintendo}introduce la consola de
videojuegos \textbf{Nintendo 64}, que utiliza el \textbf{microprocesador VR4300} de 64 bits, construido alrededor de una variante de bajo costo del
\textbf{MIPS R4000}. En 1998 \emph{IBM} lanza la línea \textbf{POWER3} de microprocesadores de 64 bits. 

\subsection{La llegada de los 64 bits a las computadoras personales}
En septiembre de 2003 \emph{AMD} introduce la arquitectura de 64 bits retrocompatible con x86, x86-64 (también llamada \textbf{AMD64}). La línea
de procesadores \textbf{Opteron} y \textbf{Athlon 64}, fue la primera basada en la arquitectura \textbf{AMD64}. En 2004 \emph{Intel} introduce la
arquitectura \textbf{EM64T}(luego fue renombrada a \textbf{Intel 64}), que es una versión de 64 bits de la arquitectura x86. La primera línea
de procesadores basados en \textbf{EM64T} fue la línea \textbf{Pentium 4}. Ambas versiones pueden ejecutar aplicaciones heredadas de 32 bits sin ninguna
penalización en el rendimiento, así como el nuevo software de 64 bits.


\begin{figure}[htb]
	\centering
	\includegraphics[scale = 0.15]{Graphics/CPU-NUS_01-Nintendo64.jpg}
	\caption{\textbf{Microprocesador VR4300} de la \textbf{Nintendo 64}}
	\label{fig:19}
\end{figure}

\begin{figure}[htb]
	\centering
	\includegraphics[scale = 0.15]{Graphics/AMD_Athlon_64_3200+_ADA3200AEP5AP.jpg}
	\caption{\textbf{AMD Athlon 64 3200}}
	\label{fig:20}
\end{figure}
\newpage
\section{Intel: el camino hacia los Intel Core}
\emph{Intel Corporation} es una corporación multinacional estadounidense y una empresa de tecnología con sede en \emph{Santa Clara, California}. 
Es el fabricante de chips semiconductores más grande del mundo por ingresos y es uno de los desarrolladores de la serie x86 
de conjuntos de instrucciones x86 \brackcite{wikipedia_2022_x86_ISA}, los conjuntos de instrucciones que se encuentran en la mayoría de las 
computadoras personales. Muchos de sus productos como el \textbf{4004} o el \textbf{8086}, vistos en las secciones anteriores, ocupan un lugar especial 
en la historia de los microprocesadores. A continuación veremos a otros de los microprocesadores y las arquitecturas que convirtieron 
a este gigante de los semiconductores en lo que es hoy en día.\

\subsection{\textbf{80186} y \textbf{80188}}
\emph{Intel} siguió al \textbf{8086} con varios otros procesadores, los cuales usaban una arquitectura similar de 16 bits. El primero fue \textbf{80186}, 
destinado a aplicaciones integradas. Para facilitar esto, \emph{Intel} integró varias piezas de hardware que normalmente se encuentran en la 
placa base en la CPU, incluido el generador de reloj, el controlador de interrupción y el temporizador. El \textbf{80188}, orientado al presupuesto, 
contenía de manera similar varias piezas de hardware integradas en el procesador. Pero al igual que el \textbf{8088}, su bus de datos se redujo a la 
mitad.

\subsection{\textbf{80286}: más memoria, más rendimiento}
El \textbf{80286}(también comercializado como \textbf{iAPX 286} y a menudo llamado \textbf{Intel 286}) se lanzó en 1982, el mismo año que el \textbf{80186} y tenía
características casi idénticas, pero ampliaba el bus de direcciones a 24 bits, lo que permitía al procesador acceder a hasta 16 MB de memoria.
El \textbf{80286} usó aproximadamente 134,000 transistores. \brackcite{sexton_2018_history_of_intel_cpus,wikipedia_2022_80286}.

\subsection{iAPX 432}
En 1981 \emph{Intel}  introduce el \textbf{iAPX 432}. Este \textbf{microprocesador} fue uno de los primeros intentos de Intel de desviarse de su cartera x86 a favor de un 
diseño completamente diferente. \emph{Intel} esperaba que \textbf{iAPX 432} fuera varias veces más rápido que sus otras ofertas. Sin embargo, el procesador 
finalmente falló debido a algunos defectos de diseño importantes. Aunque los procesadores x86 son relativamente complejos, el \textbf{iAPX 432} 
llevó CISC a un nivel completamente nuevo de complejidad\brackcite{sexton_2018_history_of_intel_cpus,wikipedia_2022_iapx_432}.

\subsection{i960: El primer microprocesador RISC de Intel}
RISC (del inglés Reduced Instruction Set Computer) es un tipo de diseño de CPU generalmente utilizado en microprocesadores o 
con las siguientes características fundamentales:
\begin{itemize}
	\item Instrucciones de tamaño fijo y presentadas en un reducido número de formatos.
	\item Solo las instrucciones de carga y almacenamiento acceden a la memoria de datos.
\end{itemize}
RISC está a favor de conjuntos de instrucciones pequeñas y simples que toman menor tiempo para ejecutarse.
Intel creó su primer procesador RISC en 1984 \brackcite{wikipedia_2022_RISC}. No fue diseñado como un competidor directo de los procesadores 
x86 de la empresa porque estaba pensado como una solución integrada segura. Internamente, era una arquitectura superescalar de 32 bits que 
usaba los conceptos de diseño de Berkeley RISC\brackcite{sexton_2018_history_of_intel_cpus}.

\subsection{80486: Integrando la FPU}
El 80486(i486) de Intel fue otro paso importante en términos de rendimiento. La clave de su éxito fue una integración más estrecha de los componentes 
en la CPU. El 80486 fue la primera CPU x86 en contener caché L1. Los primeros modelos 80486 venían con 8 KB en matriz y se grabaron en un proceso de 1000 nm.
Intel también incorporó la \textbf{FPU}(unidad de punto flotante diseñada especialmente para llevar a cabo operaciones con números de punto flotante) a la CPU, que hasta ese 
momento había sido una unidad de procesamiento funcional independiente. Al mover estas piezas de hardware al procesador host, la latencia entre ellas se redujo 
drásticamente. El 80486 también usó una interfaz FSB(front-side bus)\brackcite{wikipedia_2022_front_side_bus} más rápida para aumentar el ancho de banda, y el núcleo 
tenía varios otros ajustes para impulsar el IPC. Estos cambios aumentaron significativamente el rendimiento del 80486 y los modelos de gama alta fueron varias 
veces más rápidos que el 80386 anterior\brackcite{sexton_2018_history_of_intel_cpus, }. 
\begin{figure}[htb]
	\centering
	\includegraphics[scale = 0.15]{Graphics/Intel_80486sx.jpg}
	\caption{Intel i486}
	\label{fig:21}
\end{figure}

\subsection{Pentium}

El \textbf{Pentium} surgió en 1993 como el primer microprocesador \emph{Intel x86} que no seguía el sistema numérico 80x86. Internamente, el \textbf{Pentium} utilizó 
la arquitectura \textbf{P5}, que fue el primer diseño superescalar \emph{x86} de \emph{Intel}. Aunque el \textbf{Pentium} fue generalmente más rápido que 
el \textbf{80486} en todos los sentidos, su característica más destacada fue una \textbf{FPU} sustancialmente mejorada. La \textbf{FPU} del \textbf{Pentium} original 
era hasta 10 veces más rápida que la unidad antigua del \textbf{80486}. Esto se convirtió en una característica aún más significativa en años posteriores cuando Intel 
lanzó el \textbf{Pentium MMX}. Este microprocesador tenía la misma arquitectura que el \textbf{Pentium} original, pero incluía soporte para el nuevo conjunto de instrucciones 
MMX SIMD de \emph{Intel} que podría aumentar drásticamente el rendimiento. \emph{Intel} también aumentó el tamaño de caché L1 en sus procesadores \textbf{Pentium} en relación con el \textbf{80486}. 
Los \textbf{Pentium} iniciales contenían 16 KB, mientras que el \textbf{Pentium MMX} subió a 32 KB. Naturalmente, estos procesadores también funcionaron a velocidades de reloj más altas. 
\emph{Intel} planeó seguir rápidamente al \textbf{Pentium} con el \textbf{Pentium Pro} basado en su arquitectura \textbf{P6}, pero se encontró con dificultades técnicas. El \textbf{Pentium Pro} 
fue considerablemente más rápido que el \textbf{Pentium} en operaciones de 32 \emph{bits}, gracias a su diseño \emph{out-of-order}(paradigma en el que el microprocesador ejecuta 
instrucciones en un orden regido por la disponibilidad de datos de entrada y unidades de ejecución, en lugar de por su orden original en un programa)\brackcite{wikipedia_2022_out_of_order}. En 1997 cuando con el lanzamiento de 
\textbf{Pentium II} se logró superar casi todos los aspectos negativos del \textbf{Pentium Pro}. Su arquitectura subyacente era similar a la del \textbf{Pentium Pro}, 
y continuó usando un \emph{papeline} de 14 etapas con varias mejoras en el núcleo para mejorar el IPC(Del inglés \emph{instructions per clock} es la cantidad de instrucciones que es 
capaz de ejecutar por cada ciclo de reloj el microprocesador). \emph{Intel} planeó continuar luego  del \textbf{Pentium II} con un procesador basado en su arquitectura \emph{Netburst}, 
pero aún no estaba listo. En cambio, \emph{Intel} volvió a impulsar la arquitectura \textbf{P6} con \textbf{Pentium III}. Gracias a un \emph{pipeline} actualizado y a un 
aumento en la velocidad del reloj(el \textbf{Intel Pentium III-T Tualatin-256} llegó a alcanzar los 1.13GHz), los procesadores \textbf{Pentium III} superaron a sus 
homólogos \textbf{Pentium II}.\\ En 2000, la arquitectura \emph{Netburts} de \emph{Intel} finalmente estuvo lista y se puso en producción como \textbf{Pentium 4}. 
Los microprocesadores de \emph{Intel} llevarían esta arquitecura durante los próximos seis años. La primera implementación de esta arquitecura fue nombrada como 
\emph{Willamette}, en la cual se basaron los \textbf{Pentium 4} durante los primeros dos años. Sin embargo, este fue un momento difícil para \emph{Intel}, y el chip luchó para 
superar al \textbf{Pentium III}. \emph{Netburts} permitió frecuencias significativamente más altas y \emph{Willamette} logró alcanzar los 2 GHz, pero el \textbf{Pentium III} 
a 1,4 GHz fue aún más rápido en algunas tareas. La situación mejoró con el diseño a 130 nm de Intel conocido como \emph{Northwood}, el cual  amplió la frecuencia 
de reloj hasta los 3,2 GHz y duplicó la memoria caché L2 de 256 KB a 512 KB. \emph{Northwood} se desempeñó significativamente mejor y fue muy competitivo frente a los microprocesadores 
de \emph{AMD} de esa época.
En los modelos de gama alta, \emph{Intel} también introdujo su tecnología \textbf{Hyper-Threading}(innovación de hardware que permite que se ejecuten varios subprocesos en cada núcleo, 
lo que significa que se puede hacer más trabajo en paralelo)\brackcite{WhatIsHy83:online} para mejorar la utilización de recursos en entornos que 
enfatizaban la multitarea. \textbf{Hyper-Threading} no fue tan beneficioso en \emph{Northwood} como lo es en los procesadores \textbf{Core i7} actuales, 
pero aumentó el rendimiento en algunos puntos porcentuales.
\emph{Northwood} llevó la arquitectura \emph{Netburts} desde 2002 hasta 2004, después de lo cual \emph{Intel} lanzó \emph{Prescott} con numerosas mejoras. 
Usó un proceso de fabricación de 90 nm que permitió a \emph{Intel} aumentar la caché L2 a 1 MB.  \emph{Prescott} tenía un ancho de banda significativamente mayor que \emph{Northwood}, 
lo cual fue vital para aumentar el rendimiento de \emph{Netburts} \brackcite{sexton_2018_history_of_intel_cpus, }.




\begin{figure}[htb]
	\centering
	\includegraphics[scale = 0.15]{Graphics/Intel_Pentium_P5.jpg}
	\caption{Intel Pentium P5}
	\label{fig:22}
\end{figure}

\begin{figure}[htb]
	\centering
	\includegraphics[scale = 0.15]{Graphics/Pentium_4_-_SL5TK-3056.jpg}
	\caption{Intel Pentium 4 Willamette}
	\label{fig:23}
\end{figure}
\newpage
\subsection{Intel Core 2}
Intel finalmente renunció a su arquitectura \emph{Netburst}. La empresa se dio cuenta de que P6 aún era viable y capaz de ser eficiente y brindar 
un excelente rendimiento. Reelaboró la arquitectura con su diseño Core. El 27 de julio de 2006 se introdujo La línea de procesadores Core 2.
Intel Core 2 abarca una gama de microprocesadores Intel de consumo de 64 bits x86-64 de uno, dos y cuatro 
núcleos basados en la microarquitectura Nehalem. Los modelos de uno y dos núcleos son de un solo \emph{die}(parte donde se encuentra 
todo el circuito y los componentes internos del microprocesador), mientras que los modelos de cuatro núcleos comprenden dos \emph{dies}, cada una con dos núcleos, 
empaquetados en un módulo de varios chip(MCM). El tamaño de la memoria caché  L2 de los Core 2 iba desde los 512 KB a los 
12 MB \brackcite{wikipedia_2022_intel_core_2,sexton_2018_history_of_intel_cpus}.

\begin{figure}[htb]
	\centering
	\includegraphics[scale = 0.8]{Graphics/IntelCore_2_Q6600.jpeg}
	\caption{Intel Core 2 QuadQ6600}
	\label{fig:24}
\end{figure}

\newpage

\subsection{Intel Core: la llegada de los i3,i5,i7}
Con el mercado de procesadores en un estado altamente competitivo, Intel no pudo darse el lujo de quedarse quieto por mucho tiempo. 
Por lo tanto, modificó la arquitectura Core para crear Nehalem, que agrega numerosas mejoras. El controlador de caché se rediseñó y 
la caché L2 se redujo a 256 KB por núcleo. Sin embargo, esto no afectó el rendimiento, ya que Intel agregó entre 4 y 12 MB de caché L3 
compartidos entre todos los núcleos. Las CPU basadas en Nehalem aparecieron entre uno y cuatro núcleos. La última gran 
ventaja que tuvo Nehalem sobre la arquitectura Core fue que marcó el regreso de la tecnología Hyper-Threading. Gracias a esta y 
muchas otras mejoras, Nehalem pudo funcionar hasta el doble de rápido que los procesadores Core 2 en cargas de trabajo con muchos 
subprocesos. Nehalem funcionó a frecuencias de reloj comparables a Core 2 y tenía un proceso de fabriación de 45nm. 
Nehalem también fue el primer procesador de Intel en implementar \emph{Turbo Boost} (característica que permite aumentar automáticamente la frecuencia 
cuando el sistema operativo solicita el estado de mayor rendimiento del microprocesador). Aunque el reloj base del procesador Nehalem más rápido 
alcanzó un máximo de 3,33 GHz, podría operar a 3,6 GHz por períodos cortos gracias a esta nueva tecnología. La última gran ventaja que tuvo 
Nehalem sobre la arquitectura \emph{Core 2} fue que marcó el regreso de la tecnología \emph{Hyper-Threading}
(innovación de hardware de Intel que permite que se ejecuten múltiples subprocesos en cada núcleo, lo que significa que se puede hacer más trabajo en paralelo)\brackcite{WhatIsHy83:online}. 
Gracias a esta y muchas otras mejoras, Nehalem pudo funcionar hasta el doble de rápido que los procesadores Core 2 en cargas de trabajo con muchos subprocesos. I
ntel vendió CPU Nehalem bajo las marcas Celeron, Pentium, Core i3, Core i5, Core i7 y Xeon.

\begin{figure}[htb]
	\centering
	\includegraphics[scale = 0.2]{Graphics/Intel_core_i7_940_top_R7309478_wp.jpg}
	\caption{Intel Core i7 940  Nehalem}
	\label{fig:25}
\end{figure}



\textbf{Mejoras en las próximas generaciones:}\\
La segunda, tercera, cuarta, quinta , sexta y séptima generación de los \emph{Intel Core} dieron un salto significativo en rendimiento, mejorarando
el IPC, subiendo la frecuencia de reloj,  con   procesos de fabricación  más eficientes que llegan hasta los 14 nm en la sexta y séptima generación.
Aumentaron la memoria caché, principalmente la L3(hasta 8MB compartidos en la séptima generación). Se mejoraron los gráficos integrados y se introdujeron 
compatibilidad con nuevas tecnologías como \emph{Intel Optane} y \emph{Thunderbolt 3}.\\
Con la octava generación (\emph{Coffee Lake}), lanzada en 2017 se añadieron más núcleos. Se introdujeron microprocesadorer Core i5 y Core i7 con seis núcleos 
(junto con \emph{hyper-threading} en el caso de los i7) y microprocesadores Core i3 con cuatro núcleos y sin \emph{hyper-threading}.
La novena generación, lanzada en 2018 fue un refrescamiento de \emph{Coffe Lake}. Las principales diferencias con respecto a la 
octava generación (además del aumento de frecuencia del reloj) son:
\begin{itemize}
	\item Las Core i7 contienen 8/8 núcleos/hilos en comparación con los 6/12 que tenía los \emph{Core i7} de octava generación.
	\item Los Core i3 están equipados con tecnología \emph{Turbo Boost}. 	
\end{itemize}

Con la décima y oncena generación(Comet Lake y Rocket Lake respectivamente)  los Core i7 pasaron a tener 8 núcleos/16 hilos mientras los Core i5  
pasaron a tener 6 núcleos/12 hilos. Estas dos generaciones representaron una ligera mejora a \emph{Coffe Lake}.

\subsection{Alder Lake: la arquitecura híbrida}

\emph{Intel} anunció oficialmente la decimosegunda generación de los procesadores \textbf{Intel Core} (\emph{Alder Lake}) en 2021, 
basados en una arquitectura híbrida que utiliza núcleos de rendimiento  y núcleos eficientes.

Por primera vez \emph{Intel} ha apostado en su gama principal de procesadores por el uso de dos tipos de núcleo de diferente 
rendimiento y consumo para acelerar las tareas del procesador:

\begin{itemize}
	\item Los \textbf{P-Cores} (\emph{Golden Cove}) son núcleos de gran tamaño y que ofrecen un gran rendimiento, pero a cambio de ser menos eficientes energéticamente.
	\item Los \textbf{E-Cores} (\emph{Gracemont}) son núcleos de menor tamaño, lo que le permite a \emph{Intel} colocar una mayor cantidad en un mismo espacio y 
	tienen menor consumo energético.
\end{itemize}


Dependiendo del modelo de Intel Core 12 la cantidad de \textbf{P-Cores} y \textbf{E-Cores} varía, pero ambos trabajan al unísono y de manera coordinada. 
Para esto último hacen uso de una caché de 30 MB de tercer nivel(L3), la cual no solo le da acceso al mismo espacio de memoria 
a ambos tipos de núcleos, sino que además les permite trabajar conjuntamente en la ejecución de los procesos a ejecutar \brackcite{wikipedia_2022_alder_lake,roca_2021_alder_lake}.

\begin{figure}[htb]
	\centering
	\includegraphics[scale = 0.2]{Graphics/Intel_CPU_Core_i7_12700K_Alder_Lake.jpg}
	\caption{Intel Core i7 12700K Alder Lake}
	\label{fig:26}
\end{figure}

\section{AMD: El camino hacia los Ryzen}
\section{Eficiencia por watt y la autonomía de los microprocesadores}
