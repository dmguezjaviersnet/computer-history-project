\chapter{El microprocesador}\label{chapter:microprocesador}

El \textbf{microprocesador} es un procesador de computadora, donde la lógica del procesamiento de datos y el control
están incluidos en un solo circuito integrado, o en un pequeño número de circuitos integrados. El \textbf{microprocesador}
contiene los circuitos aritméticos, lógicos y de control necesarios para realizar las funciones de la unidad central de
procesamiento  de una computadora. El circuito integrado es capaz de interpretar y ejecutar instrucciones de programa y
realizar operaciones aritméticas. El \textbf{microprocesador} es un circuito integrado digital multipropósito, controlado
por reloj y basado en registros que acepta datos binarios como entrada, los procesa de acuerdo con las instrucciones
almacenadas en su memoria y proporciona resultados (también en forma binaria) como salida. Un \textbf{microprocesador}
hipotético mínimamente funcional podría incluir solo una \textbf{ALU}(\emph{Arithmetic Logic Unit}) y una sección lógica
de control. La \textbf{ALU} realiza sumas, restas y operaciones como \textbf{AND} y \textbf{OR}. Cada operación de la
\textbf{ALU} establece uno o más \emph{flags} en un registro de estados, que indican los resultados de la última operación
(por ejemplo, si el resultado es cero, si es negativo, si hay desbordamiento, etc.). La lógica de control recupera los
código de instrucción desde la memoria e inicia la secuencia de operaciones necesarias parar que la \textbf{ALU} lleve a
cabo la instrucción \brackcite{wikipedia_2022_Microprocesador}.\\ Antes de los microprocesadores las computadoras pequeñas
se construían utilizando  \emph{racks} de placas de circuito con muchos circuitos integrados de mediana y pequeña escala,
generalmente del tipo \textbf{TTL}\emph{(Transistor-Transistor Logic)} \brackcite{wikipedia_2022_ttl}. Los microprocesadores
combinaron esto en uno o unos pocos circuitos integrados a gran escala.\\ El incremento continuo de las capacidades de los
microprocesadores desde  que se empezaron a fabricar ha dejado otras formas de computadoras casi complemtamente obsoletas,
con uno o más microprocesadores usados en todo desde pequeños sistemas embebidos y dispostivos portátiles hasta los enormes
\emph{mainframes} y las supercomputadoras

\section{Surgimiento del microprocesadaor}
Las invenciones a veces ocurren cuando las personas se enfrentan a un problema y luchan por resolverlo. En otras ocasiones, suceden 
cuando las personas adoptan una meta visionaria. La historia de cómo \emph{Ted Hoff} y su equipo de \emph{Intel} inventaron el \textbf
{microprocesador} es un caso de ambos \brackcite{isaacson_2014}. \emph{Hoff}, que había sido un joven profesor en \emph{Stanford}, se
convirtió en el duodécimo empleado de \emph{Intel}, donde fue asignado para trabajar en el diseño de chips. \emph{Hoff} se dió cuenta de que era un
desperdicio y poco elegante diseñar muchos tipos de microchips donde cada uno tuviera una función diferente,  que era lo que \textbf
{Intel} estaba haciendo hasta ahora. Llegaría una empresa y le pediría que construyera un \textbf{microchip} diseñado para realizar una tarea
específica. \emph{Hoff} imaginó, al igual que \emph{Noyce} y otros, un enfoque alternativo: crear un \textbf{chip} de propósito general que pudiera
ser instruido o programado para realizar una variedad de aplicaciones diferentes según se desee. En otras palabras, una computadora de propósito
general en un \textbf{chip}. Esta visión coincidió con un problema que se planteó a \emph{Intel} en el verano de 1969. Una empresa japonesa llamada \emph
{Busicom} estaba planeando una nueva y poderosa calculadora de escritorio : la \textbf{Busicom 141-PF}, y había elaborado especificaciones para
doce microchips de propósito especial para diseñar 12 chips personalizados para su nueva calculadora de impresión (diferentes para
manejar pantalla, cálculos, memoria, etc.) que quería que \emph{Intel} construyera. \emph{Intel} estuvo de acuerdo y se fijó un precio. \emph{Noyce} le
pidió a \emph{Hoff} que supervisara el proyecto. La cantidad de chips y su complejidad fue mucho mayor de lo que esperaba, por lo que no
había forma de que \emph{Intel} pudidiera construirlos al precio acordado. \emph{Intel} se propuso diseñar un solo \textbf{chip} lógico
que pudiera realizar casi todas las tareas que quería \emph{Busicom}. Para empeorar las cosas,  la creciente popularidad de la calculadora
de bolsillo de \emph{Jack Kilby} estaba obligando a \emph{Busicom} a reducir aún más su precio. \emph{Noyce} le dijo a \emph{Intel} que lo intentara y  tuvo que
convencer a Grove del proyecto antes de venderle la idea a \emph{Busicom}. En septiembre de 1969, \emph{Intel} y su colega \emph{Stan Mazor} habían esbozado
la arquitectura de un \textbf{chip} lógico de uso general que podía seguir instrucciones de programación. Sería capaz de hacer el trabajo de
nueve de los doce chips que había solicitado \emph{Busicom}. Se diseñaron un conjunto cuatro chips conocido como \textbf{MCS-4}. Incluía un \textbf{chip} de
unidad de procesamiento central (\textbf{CPU}), el \textbf{4004}, así como un \textbf{chip} de memoria de solo lectura (\textbf{ROM}) compatible para los programas de
aplicaciones personalizadas, un \textbf{chip} de memoria de acceso aleatorio (\textbf{RAM}) para procesar datos, y un \textbf{chip} de registro para el puerto
de entrada/salida (E/S). Cuando llegó el momento de renegociar el precio, \emph{Intel} hizo una recomendación fundamental a \emph{Noyce}, que ayudó
a crear un gran mercado para los chips de uso general y aseguró que \emph{Intel} seguiría siendo un impulsor de la era digital. A cambio de
darle a \emph{Busicom} un buen precio, \emph{Noyce} insistió en que \emph{Intel} retuviera los derechos del nuevo \textbf{chip} y se le permitiera licenciarlo a otras 
compañías para fines distintos a la fabricación de una calculadora. Debido a que era esencialmente un procesador de computadora en un \textbf{chip},
el nuevo dispositivo se denominó \textbf{microprocesador}. En noviembre de 1971 \emph{Intel} dio a conocer el producto, el \textbf{Intel 4004}, al público.
Este tenía un precio de salida de \$200 \brackcite{intel_4004, wikipedia_2022_Microprocesador,campbell-kelly_garcia-swartz_2015,isaacson_2014}.

\begin{figure}[htb]
	\centering
	\includegraphics[scale = 0.25]{Graphics/faggin_hoff_mazor_-4004.jpg}
	\caption{Desde la izquierda, \emph{Federico Faggin}, \emph{Ted Hoff} y \emph{Stanley Mazor} con procesadores \textbf{Intel 4004}}
	\label{fig:10}
\end{figure}

\subsection{Intel 4004}
Este revolucionario \textbf{microprocesador}, del tamaño de la uña del dedo meñique, entregaba la misma potencia de cómputo que la primera
computadora electrónica construida en 1946, que llenó una habitación entera. El primer \textbf{microprocesador} \textbf{Intel 4004} se
fabricó en obleas de dos pulgadas, en comparación con las obleas de 12 pulgadas que se utilizan habitualmente en los productos actuales.
El \textbf {microprocesador} \textbf{Intel 4004} es único en el sentido de que es uno de los diseños de \textbf{microprocesador} más
pequeños que alguna vez entró en producción comercial. El \textbf{4004}  era un \textbf{microprocesador} de \textbf{4-bits}, alcanzaba una máxima
velocida de reloj 740 kHz. El ancho de línea del circuito del \textbf{microprocesador} \textbf{Intel 4004} era de 10 micrones o 10 000
nanómetros. En 1971 el \textbf{Intel 4004} contenía 2300 transistores Para el 2010, un procesador \textbf{Intel Core} con matriz de procesamiento
de 32 nm y tecnología de \textbf{silicio} de puerta metálica de alta k de segunda generación contenía  560 millones de transistores. En comparación,
un cabello humano promedio tiene 100,000 nanómetros de ancho \brackcite{intel_4004,wikipedia_2022_intel_4004}.      

\begin{figure}[htb]
	\centering
	\includegraphics[scale = 0.15]{Graphics/Intel_C4004.jpg}
	\caption{Primer \textbf{microprocesador}: \textbf{Intel 4004}}
	\label{fig:12}
\end{figure}


\section{El arribo del primer microprocesador de 8-bits}
El \textbf{Intel 4004} fue seguido en 1972 por el \textbf{Intel 8008}, el primer \textbf{microprocesador} de 8 bits del mundo.
Sin embargo, el \textbf{8008} no fue una extensión del diseño del \textbf{4004}, sino la culminación de un proyecto de diseño separado
en \emph{Intel}, que surgió de un contrato con \emph{Computer Terminals Corporation}(\textbf{CTC}), por un \textbf{chip} para
una terminal que estaban diseñando: el \textbf{Datapoint 2200}, los aspectos fundamentales del diseño no provinieron de \emph{Intel}
sino de \textbf{CTC}. El \textbf{8008} salió con 3.500 transistores, tenía un bus de datos de 8 bits y un bus de direcciones de 14 bits. Este
\textbf{micoprocesador} fue la base del famoso kit de computadora "Mark-8" \brackcite{wikipedia_2022_intel_8008, campbell-kelly_garcia-swartz_2015}.

\begin{figure}[htb]
	\centering
	\includegraphics[scale = 0.15]{Graphics/Intel_C8008-1.jpg}
	\caption{\textbf{Microprocesador}  \textbf{Intel 8008}}
	\label{fig:13}
\end{figure}

\subsection{8080}
El Intel \textbf{8080} es el segundo \textbf{microprocesador} de 8 bits diseñado y fabricado por Intel. Apareció por primera vez en abril de 1974 y es una variante ampliada y mejorada
del diseño anterior del \textbf{8008}, aunque sin compatibilidad binaria. La velocidad de reloj o límite de frecuencia especificado inicialmente era de 2 MHz, con instrucciones
comunes que usaban 4, 5, 7, 10 o 11 ciclos. Como resultado, el procesador podía ejecutar varios cientos de miles de instrucciones por segundo. Federico Faggin lo concibió 
y diseñó utilizando MOS de canal N de alto voltaje.

\begin{figure}[htb]
	\centering
	\includegraphics[scale = 0.2]{Graphics/8080_microprocessorr.jpg}
	\caption{\textbf{Microprocesador} \textbf{Intel 8080}}
	\label{fig:14}
\end{figure}

\section{El arribo de los 16 bits}

El primer \textbf{microprocesador} multichip de 16 bits fue el \textbf{IMP-16} de \emph{National Semiconductor}, presentado a principios de 1973. Una versión de 8 bits del conjunto de 
chips se introdujo en 1974 como \textbf{IMP-8}. Durante el mismo año, \emph{National} presentó el primer \textbf{microprocesador} de un solo \textbf{chip} de 16 bits, el \textbf{PACE},
seguido de una versión NMOS, el \textbf{INS8900}. El primer \textbf{microprocesador} de 16 bits de un solo chip fue el TMS 9900 de \emph{Texas Instruments}, presentado en 1976, que
también era compatible con su línea de minicomputadoras \textbf{TI-990}. \emph{Intel} produjo su primer procesador de 16 bits, el \textbf{8086}, en 1978. Era compatible con el \textbf{8080}
y el \textbf{8085} (un derivado del \textbf{8080}) \brackcite{wikipedia_2022_Microprocesador,staff_2021_microprocessor}.

\begin{figure}[htb]
	\centering
	\includegraphics[scale = 0.2]{Graphics/NSIMP-16A.jpg}
	\caption{\textbf{Microprocesador} \textbf{IMP-16}}
	\label{fig:15}
\end{figure}

\section{8086: El comienzo de x86}
El primer procesador de 16 bits de \emph{Intel} fue el \textbf{8086}, que ayudó a mejorar considerablemente el rendimiento en comparación con los diseños anteriores.
Este \textbf{microprocesador} utilizaba la misma microarquitectura que los microprocesadores de 8 bits de Intel (\textbf{8008}, \textbf{8080} y \textbf{8085}). Esto permitió
que los programas en lenguaje ensamblador escritos en 8 bits migraran sin problemas. El \textbf{8086} no solo tenía una frecuencia más alta que la \textbf{8088}, sino que
también empleaba un bus de datos externo de 16 bits y una cola de precarga de 6 bytes más larga. También podía ejecutar tareas de 16 bits (aunque la mayoría del \emph{software} 
en ese momento estaba diseñado para procesadores de 8 bits). El bus de direcciones se amplió a 20 bits, lo que permitió al \textbf{8086} acceder a hasta 1 MB de memoria y,
por lo tanto, aumentar el rendimiento. El \textbf{8086} también se convirtió en el primer procesador x86 y utilizó la primera revisión del x86 ISA \brackcite{wikipedia_2022_x86_ISA},
en el que se han basado casi todos los procesadores creados por \emph{AMD} o \emph{Intel} desde la introducción del \textbf{8086} 
\brackcite{wikipedia_2022_Microprocesador, wikipedia_2022_intel_8086,sexton_2018_history_of_intel_cpus}.

\begin{figure}[htb]
	\centering
	\includegraphics[scale = 0.2]{Graphics/Intel_C8086.jpg}
	\caption{\textbf{Intel 8086}}
	\label{fig:16}
\end{figure}

\section{El arribo de los 32 bits}

Los diseños de 16 bits solo habían estado en el mercado brevemente cuando comenzaron a aparecer implementaciones de 32 bits.
El más significativo de los diseños de  microprocesadores de 32 bits fue el Motorola MC68000, introducido en 1979. 
El 68k, como se le conocía ampliamente, tenía  registros de 32 bits en su modelo de programación pero usaba rutas internas
de datos de 16 bits, tres unidades aritméticas-lógicas de 16 bits, y un bus de datos externo de 16 bits para reducir 
el conteo de pines. Soportaba externamente solo direcciones de 24 bits (internamente funcionaba con direcciones completas de 
32 bits). Los diseños de Apple Lisa y Macintosh utilizaron el 68000, al igual que muchos otros diseños a mediados de la década de 
1980, incluidos Atari ST y Commodore Amiga. El primer microprocesador de 32 bits de un solo chip del mundo, con rutas de datos de 
32 bits, buses de 32 bits y direcciones de 32 bits, fue el AT\&T Bell Labs BELLMAC-32A, con las primeras muestras en 1980 y la producción general en 1982 . 
Después de la venta de AT\&T en 1984, pasó a llamarse WE 32000 (WE de Western Electric) y tuvo dos generaciones posteriores, WE 32100 y WE 32200.
El primer microprocesador comercial de un solo chip de 32 bits disponible en el mercado fue el HP FOCUS.
El primer microprocesador de 32 bits de Intel fue el iAPX 432, que se introdujo en 1981, pero no fue un éxito comercial. Tenía una arquitectura orientada 
a objetos basada en capacidades avanzadas, pero un rendimiento deficiente en comparación con las arquitecturas contemporáneas, como la 80286 de Intel 
(introducida en 1982), que era casi cuatro veces más rápida en las pruebas de referencia típicas. Sin embargo, los resultados para el iAPX432 se 
debieron en parte a un compilador de Ada apresurado y, por lo tanto, subóptimo.

\begin{figure}[htb]
	\centering
	\includegraphics[scale = 0.15]{Graphics/Motorola_MC68HC000FN8-0695.jpg}
	\caption{Motorola 68K}
	\label{fig:17}
\end{figure}
\subsection{80386: x86 se convierte a 32-bit}
El primer procesador x86 de 32 bits de Intel fue el 80386(renombrado más tarde como i386), lanzado en 1985. Una ventaja clave que tenía este procesador era su bus de 
direcciones de 32 bits que le permitía admitir hasta 4 GB de memoria del sistema. Aunque esto era mucho más de lo que nadie estaba 
usando en ese momento, las limitaciones de RAM a menudo perjudican el rendimiento de los procesadores x86 anteriores y de la competencia.
\begin{figure}[htb]
	\centering
	\includegraphics[scale = 0.1]{Graphics/Intel_i386DX.jpg}
	\caption{Intel i386}
	\label{fig:18}
\end{figure}
\newpage
\section{El arribo de los 64 bits}
Antes que se introdujeran los primeros microprocesadores de 64 bits destinados a las computadoras peronsales, desde principios de la década de 1990 se 
utilizaron diseños de microprocesadores de 64 bits en varios mercados. En 1991 MIPS Computer Systems produce el primer microprocesador de 64 bits, el R4000, 
que implementa la arquitectura MIPS III, la tercera revisión de su arquitectura MIPS. El R4000 se utilizó en estaciones de trabajo de gráficos SGI. En 1996
la compañía nipona de videojuegos Nintendo introduce la consola de videojuegos Nintendo 64, que utiliza el microprocesador VR4300 de 64 bits, construido alrededor 
de una variante de bajo costo del MIPS R4000. En 1998 IBM lanza la línea POWER3 de microprocesadores de 64 bits \brackcite{wikipedia_2022_Microprocesador,wikipedia_2022_motorla_6800}.
\subsection{La llegada de los 64 bits a las computadoras personales}
EN septiembre de 2003 AMD introduce  la arquitectura de 64 bits retrocompatible con x86, x86-64 (también llamada AMD64). La línea de procesadores Opteron 
y Athlon 64, fue la primera basada en la arquitectura AMD64. En 2004 Intel introduce la arquitectura EM64T(luego fue renombrada a Intel 64), que es una versión de 64 bits 
de la arquitectura x86. La primera línea de procesadores basados en EM64T fue la línea Pentium 4.  Ambas versiones pueden ejecutar aplicaciones heredadas de 32 bits sin 
ninguna penalización en el rendimiento, así como el nuevo software de 64 bits \brackcite{wikipedia_2022_Microprocesador}.


\begin{figure}[htb]
	\centering
	\includegraphics[scale = 0.15]{Graphics/CPU-NUS_01-Nintendo64.jpg}
	\caption{Microprocesador VR4300 de la Nintendo 64}
	\label{fig:19}
\end{figure}

\begin{figure}[htb]
	\centering
	\includegraphics[scale = 0.15]{Graphics/AMD_Athlon_64_3200+_ADA3200AEP5AP.jpg}
	\caption{AMD Athlon 64 3200}
	\label{fig:20}
\end{figure}

\section{Intel: el camino hacia los Intel Core}
Intel Corporation es una corporación multinacional estadounidense y una empresa de tecnología con sede en Santa Clara, California. 
Es el fabricante de chips semiconductores más grande del mundo por ingresos y es uno de los desarrolladores de la serie x86 
de conjuntos de instrucciones x86\brackcite{wikipedia_2022_x86_ISA}, los conjuntos de instrucciones que se encuentran en la mayoría de las 
computadoras personales. Muchos de sus productos como el 4004 o el 8086, vistos en las secciones anteriores, ocupan un lugar especial 
en la historia de los microprocesadores. A continuación veremos a otros  de los microprocesadores y las arquitecturas que convirtieron 
al gigante de los semiconductores en lo que es hoy en día.\

\subsection{80186 y 80188}
Intel siguió al 8086 con varios otros procesadores, los cuales usaban una arquitectura similar de 16 bits. El primero fue 80186, 
destinado a aplicaciones integradas. Para facilitar esto, Intel integró varias piezas de hardware que normalmente se encuentran en la 
placa base en la CPU, incluido el generador de reloj, el controlador de interrupción y el temporizador. El 80188, orientado al presupuesto, 
contenía de manera similar varias piezas de hardware integradas en el procesador. Pero al igual que el 8088, su bus de datos se redujo a la 
mitad.
\subsection{80286: más memoria, más rendimiento}

El 80286(también comercializado como iAPX 286 y a menudo llamado Intel 286) se lanzó en 1982, el mismo año que el 80186 y tenía características 
casi idénticas, pero ampliaba el bus de direcciones a 24 bits, lo que permitía al procesador acceder a hasta 16 MB de memoria.
El 80286 usó aproximadamente 134,000 transistores. 
\brackcite{sexton_2018_history_of_intel_cpus,wikipedia_2022_80286}.

\subsection{iAPX 432}
En 1981 Intel introduce el iAPX 432. Este microprocesador fue uno de los primeros intentos de Intel de desviarse de su cartera x86 a favor de un 
diseño completamente diferente. Intel esperaba que iAPX 432 fuera varias veces más rápido que sus otras ofertas. Sin embargo, el procesador 
finalmente falló debido a algunos defectos de diseño importantes. Aunque los procesadores x86 son relativamente complejos, el iAPX 432 
llevó CISC a un nivel completamente nuevo de complejidad\brackcite{sexton_2018_history_of_intel_cpus,wikipedia_2022_iapx_432}.

\subsection{i960: El primer microprocesador RISC de Intel}
RISC (del inglés Reduced Instruction Set Computer) es un tipo de diseño de CPU generalmente utilizado en microprocesadores o 
con las siguientes características fundamentales:
\begin{itemize}
	\item Instrucciones de tamaño fijo y presentadas en un reducido número de formatos.
	\item Solo las instrucciones de carga y almacenamiento acceden a la memoria de datos.
\end{itemize}
RISC está a favor de conjuntos de instrucciones pequeñas y simples que toman menor tiempo para ejecutarse.
Intel creó su primer procesador RISC en 1984 \brackcite{wikipedia_2022_RISC}. No fue diseñado como un competidor directo de los procesadores 
x86 de la empresa porque estaba pensado como una solución integrada segura. Internamente, era una arquitectura superescalar de 32 bits que 
usaba los conceptos de diseño de Berkeley RISC\brackcite{sexton_2018_history_of_intel_cpus}.

\subsection{80486: Integrando la FPU}
El 80486(i486) de Intel fue otro paso importante en términos de rendimiento. La clave de su éxito fue una integración más estrecha de los componentes 
en la CPU. El 80486 fue la primera CPU x86 en contener caché L1. Los primeros modelos 80486 venían con 8 KB en matriz y se grabaron en un proceso de 1000 nm.
Intel también incorporó la FPU(unidad de punto flotante diseñada especialmente para llevar a cabo operaciones con números de punto flotante) a la CPU, que hasta ese 
momento había sido una unidad de procesamiento funcional independiente. Al mover estas piezas de hardware al procesador host, la latencia entre ellas se redujo 
drásticamente. El 80486 también usó una interfaz FSB(front-side bus)\brackcite{wikipedia_2022_front_side_bus} más rápida para aumentar el ancho de banda, y el núcleo 
tenía varios otros ajustes para impulsar el IPC. Estos cambios aumentaron significativamente el rendimiento del 80486 y los modelos de gama alta fueron varias 
veces más rápidos que el 80386 anterior\brackcite{sexton_2018_history_of_intel_cpus, }. 
\begin{figure}[htb]
	\centering
	\includegraphics[scale = 0.15]{Graphics/Intel_80486sx.jpg}
	\caption{Intel i486}
	\label{fig:21}
\end{figure}

\subsection{P5: El primer Pentium}
El Pentium surgió en 1993 como el primer procesador Intel x86 que no seguía el sistema numérico 80x86. Internamente, el Pentium utilizó la arquitectura P5, que fue el 
primer diseño superescalar x86 de Intel. Aunque el Pentium fue generalmente más rápido que el 80486 en todos los sentidos, su característica más destacada fue una FPU 
sustancialmente mejorada. La FPU del Pentium original era más de diez veces más rápida que la unidad antigua del 80486. Esto se convirtió en una característica aún más 
significativa en años posteriores cuando Intel lanzó el Pentium MMX. Este procesador tenía la misma arquitectura que el Pentium original, pero incluía soporte para el 
nuevo conjunto de instrucciones MMX SIMD de Intel que podría aumentar drásticamente el rendimiento. Intel también aumentó el tamaño de caché L1 en sus procesadores Pentium 
en relación con el 80486. Los Pentium iniciales contenían 16 KB, mientras que el Pentium MMX subió a 32 KB. Naturalmente, estos procesadores también funcionaron a 
velocidades de reloj más altas.
\begin{figure}[htb]
	\centering
	\includegraphics[scale = 0.15]{Graphics/Intel_Pentium_P5.jpg}
	\caption{Intel Pentium P5}
	\label{fig:22}
\end{figure}

\subsection{P6: El Pentium Pro}
Intel planeó seguir rápidamente al Pentium con el Pentium Pro basado en su arquitectura P6, pero se encontró con dificultades técnicas. El Pentium Pro fue considerablemente 
más rápido que el Pentium en operaciones de 32 bits, gracias a su diseño \emph{out-of-order}.  El Pentium Pro presentaba una arquitectura interna significativamente rediseñada 
que decodificaba instrucciones en microoperaciones,
\subsection{Pentium II}
Intel no abandonó la arquitectura P6, sino que esperó hasta 1997 cuando lanzó el Pentium II. El Pentium II logró superar casi todos los aspectos negativos del Pentium Pro. 
Su arquitectura subyacente era similar a la del Pentium Pro, y continuó usando un \emph{papeline} de 14 etapas con varias mejoras en el núcleo para mejorar el IPC
(Del inglés instrucction per clock es la cantidad de instrucciones que es capaz de ejecutar por cada ciclo de reloj).


\section{AMD: El camino hacia los Ryzen}
\section{Eficiencia por watt y la autonomía de los microprocesadores}



% \begin{figure}
% 	\centering
% 	\begin{subfigure}[b]
% 		\centering
% 		\includegraphics[scale = 0.1]{Graphics/CPU-NUS_01-Nintendo64.jpg}
% 		\caption{Microprocesador VR4300 de la Nintendo 64}
% 		\label{Microprocesador VR4300 de la Nintendo 64}
% 	\end{subfigure}
% 	\hfill
% 	\begin{subfigure}[b]
% 		\centering
% 		\includegraphics[scale = 0.1]{Graphics/AMD_Athlon_64_3200+_ADA3200AEP5AP.jpg}
% 		\caption{AMD Athlon 64 3200}
% 		\label{AMD Athlon 64 3200}
% 	\end{subfigure}
% 	\caption{Three simple graphs}
% 	\label{fig:18}
% \end{figure}


% En 1968, Vic Poor y Harry Pyle de CTC 
% desarrollaron el diseño original para el conjunto de instrucciones y el funcionamiento del procesador. En 1969, CTC contrató a dos empresas, Intel y Texas Instruments, para 
% realizar una implementación de un solo chip, conocida como CTC 1201. A fines de 1970 o principios de 1971, TI se retiró al no poder fabricar una pieza confiable. 
% En 1970, con Intel aún por entregar la parte, CTC optó por usar su propia implementación en el Datapoint 2200, usando en su lugar la lógica TTL tradicional 
% (por lo tanto, la primera máquina que ejecutó el "código 8008" no era de hecho un microprocesador y se entregó el año anterior). La versión de Intel del microprocesador 
% 1201 llegó a fines de 1971, pero fue demasiado tarde, lenta y requirió una cantidad de chips de soporte adicionales. CTC no tenía interés en usarlo. CTC había contratado
% originalmente a Intel por el chip y les habría debido 50 000 dólares estadounidenses (equivalente a 334 552 dólares en 2021) por su trabajo de diseño.[43] 
% Para evitar pagar por un chip que no querían (y no podían usar), CTC liberó a Intel de su contrato y les permitió el uso gratuito del diseño.[43] Intel 
% lo comercializó como 8008 en abril de 1972, como el primer microprocesador de 8 bits del mundo. Fue la base del famoso kit de computadora "Mark-8". 

% En 1973, Intel lanzó el primer procesador de 8 bits ampliamente utilizado, llamado Intel 8008. El 8008 salió con 3.500 transistores. . En este sistema, el bus de datos y 
% las direcciones eran unidades de 8 bits. El Intel 4004 fue seguido en 1972 por el Intel 8008, el primer microprocesador de 8 bits del mundo. Sin embargo, el 8008 no fue 
% una extensión del diseño del 4004, sino la culminación de un proyecto de diseño separado en Intel, que surgió de un contrato con Computer Terminals Corporation, de San Antonio TX, 
% por un chip para una terminal que estaban diseñando, [42] el Datapoint 2200: los aspectos fundamentales del diseño no provinieron de Intel sino de CTC.



 % Pronto surgió un desafío. “Cuanto más aprendía sobre este diseño, más me 
% preocupaba que Intel pudiera haber emprendido más de lo que estaba preparado para entregar”, recordó Hoff. "La cantidad de chips y su 
% complejidad fue mucho mayor de lo que esperaba". No había forma de que Intel pudiera construirlos al precio acordado. .
% Hoff propuso que Intel diseñara un solo chip lógico que pudiera realizar casi todas las tareas que quería Busicom

% “Bueno, si hay algo que se te ocurra para simplificar el diseño, ¿por qué no lo persigues?”, sugirió Noyce.
% . 
% “Sé que esto se puede hacer”, dijo sobre el chip de propósito general. “Se puede hacer para emular una computadora”. 
% Noyce le dijo que lo intentara. 
% Antes de que pudieran vender la idea a Busicom, Noyce se dio cuenta de que tenía que convencer a alguien que podría oponerse 
% aún más: Andy Grove, quien nominalmente trabajaba para él. Parte de lo que Grove vio como su mandato era mantener a Intel enfocado.

% Noyce diría que sí a casi cualquier cosa; El trabajo de Grove era decir que no. Cuando Noyce se acercó al espacio de trabajo de Grove y se 
% sentó en la esquina de su escritorio, Grove se puso inmediatamente en guardia. Sabía que el esfuerzo de Noyce por parecer indiferente era una 
% señal de que algo estaba pasando. “Estamos comenzando otro proyecto”, dijo Noyce, fingiendo reír.53 La primera reacción de Grove fue decirle a 
% Noyce que estaba loco. Intel era una empresa incipiente que todavía luchaba por fabricar sus chips de memoria y no necesitaba distracciones. Pero 
% después de escuchar a Noyce describir la idea de Hoff, Grove se dio cuenta de que la resistencia probablemente estaba mal y definitivamente era inútil.
% En septiembre de 1969, Hoff y su colega Stan Mazor habían esbozado la arquitectura de un chip lógico de uso general que podía seguir instrucciones de programación. 
%  Noyce y Hoff presentaron la opción a los ejecutivos de Busicom, 
% quienes acordaron que era el mejor enfoque.

% .  Fue un punto de negociación que Bill Gates y Microsoft emularían con IBM una década después. 
% A cambio de darle a Busicom un buen precio, Noyce insistió en que Intel retuviera los derechos del nuevo chip y se le permitiera licenciarlo a otras compañías 
% para fines distintos a la fabricación de una calculadora. 

% Se dio cuenta de que un chip que pudiera programarse para realizar cualquier función lógica se convertiría en un componente estándar en los dispositivos electrónicos, 
% de la misma manera que las piezas de madera de dos por cuatro eran un componente estándar en la construcción de casas.
% Reemplazaría los chips personalizados, lo que significaba que podría fabricarse a granel y, por lo tanto, su precio disminuiría continuamente. También marcaría el 
% comienzo de un cambio más sutil en la industria electrónica: la importancia de los ingenieros de hardware, que diseñaron la 

% ubicación de los componentes en una placa de circuito, comenzó a ser suplantada por una nueva generación, los ingenieros de software, cuyo trabajo era programar 
% Debido a que era esencialmente un procesador de computadora en un chip, el nuevo dispositivo se denominó microprocesador. En noviembre de 1971 Intel dio a conocer 
% el producto, el Intel 4004, al público. Sacó anuncios en revistas especializadas que anunciaban “una nueva era de electrónica integrada: ¡una computadora microprogramable en 
% un chip!” Tenía un precio de \$ 200 y los pedidos, así como miles de solicitudes del manual, comenzaron a llegar. Noyce asistía a una exhibición de computadoras en 
% Las Vegas el día del anuncio y estaba emocionado de ver a los clientes potenciales abarrotar la suite de Intel.

% 1969: La misión
% En 1969, Nippon Calculating Machine Corporation se acercó a Intel para diseñar 12 chips personalizados para su nueva calculadora de impresión Busicom 141-PF*. 
% Los ingenieros de Intel sugirieron una familia de solo cuatro chips, incluido uno que podría programarse para su uso en una variedad de productos, poniendo en 
% marcha una hazaña de ingeniería que alteró drásticamente el curso de la electrónica.


% 1971: era de la electrónica integrada
% Intel compró los derechos de Nippon Calculating Machine Corporation y lanzó el procesador Intel® 4004 y su conjunto de chips con un anuncio en la edición del 15 de noviembre 
% de 1971 de Electronic News: "Anunciando una nueva era en la electrónica integrada".
% Fue entonces cuando Intel® 4004 se convirtió en el primer procesador programable de uso general del mercado: un "bloque de construcción" que los ingenieros podían comprar y 
% luego personalizar con software para realizar diferentes funciones en una amplia variedad de dispositivos electrónicos.

% oderosamente pequeño, incluso en 1971
% Este microprocesador revolucionario, del tamaño de la uña del dedo meñique, entregaba la misma potencia de cómputo que la primera computadora electrónica construida en 
% 1946, que llenó una habitación entera.



