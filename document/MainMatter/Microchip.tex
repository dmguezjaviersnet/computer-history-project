\chapter{El microchip}\label{chapter:microchip}
Tiempo después \emph{Shockley} dejó los \emph{Laboratorios Bell}, y en la gala anual de la \emph
{Cámara de Comercio de Los Ángeles} donde fue galardonado por su invención, conoció a \emph{Arnold
Beckman} para el cual comenzó a trabajar en la división \emph{Laboratorio de semiconductores de Shockle}, 
parte de su empresa\emph{Beckman Instruments}, que se ubicó en \emph{Palo Alto}. En ese momento \emph{Shockley}
intentó reclutar a algunos de los investigadores que trabajaron con él en los \emph{Laboratorios Bell}, pero lo
conocían muy bien, su personalidad y su ego los espantaron. Por eso se dedicó a escribir una lista de los mejores
ingenieros de semiconductores del país. Sus más importantes contrataciones fueron \emph{Robert Noyce} doctor en 
física del \emph{Massachussets Institute of Technology}(\textbf{MIT}) y el químico \emph{Gordon Moore}. Con el paso 
del tiempo muchos de los expertos que contrató \emph{Shockley} se dieron cuenta de que era un líder incompetente, 
que cuando se equivocaba al tomar una decisión buscaba a alguien más a quien culpar, además, su indisposición a 
compartir el crédito le imposibilitó crear un espíritu de colaboración entre los hombres bajo su mando. De esta forma,
con el paso del tiempo la situación se hizo insostenible, hasta el punto que un grupo de trabajadores le dijeron a 
\emph{Arnold Beckman} que si \emph{Shockley} seguía, ellos renunciarían. Y así lo hicieron, ante la voluntad de \emph
{Beckman} de mantener a \emph{Shockley} a cargo de aquella división. Entre ellos estaban \emph{Robert Noyce} y \emph
{Gordon Moore}, y todos tenían el objetivo de crear una empresa que rivalizara con \emph{Beckman Instruments}. Para eso
eran necesarios fondos de los que no disponían, y luego de pasar mucho tiempo en búsqueda de una persona dispuesta a 
invertir en el negocio de los semiconductores, alguien sugirió ir a ver a \emph{Sherman Fairchild}, quien dispuso de
\$1.5 millones para la creación de la compañía. \\
En un artículo escrito en conmemoración del décimo aniversario del \textbf{transistor}, publicado en 1957 justo
cuando se formó \emph{Fairchild Semiconductor} y el satélite ruso \emph {Sputnik} había sido lanzado, un ejecutivo
de los \emph{Laboratorios Bell} identificó un problema que apodado "la tiranía de los números". Como el número de
componentes en un circuito aumentó, el número de conexiones aumentó mucho más rápido. Si un sistema tenía, por
ejemplo, diez mil componentes, harían falta 100.000 o más pequeños enlaces de cables en las placas de circuitos,
la mayoría de las veces soldado a mano. Esta no era una receta acequible, sin embargo era una pieza fundamental
para un nuevo gran descubrimiento. La necesidad de resolver un problema que iba empeorando coincidió muchos pequeños
avances en formas para fabricar semiconductores. Esta combinación produjo un invento que ocurrió de forma independiente
en dos lugares diferentes, \emph{Texas Instruments} y \emph{Fairchild Semiconductor}.\\
