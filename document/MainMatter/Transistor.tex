\chapter{El transistor}\label{chapter:transistor}

Antes de la invención del primer transistor a finales de la década de 1930, ya existían dispositivos que cumplían la misma función que 
este pero con una menor eficiencia.\\
\indent A finales del siglo XIX con el incipiente desarrollo de la tecnología de comunicación inalámbrica y la construcción de sistemas que
utilizaran este método por la \textbf{compañía Marconi}, \emph{Guglielmo Marconi} le asignó el cargo de consejero científico al físico
inglés \emph{John Ambrose Fleming}. \emph{Marconi} necesitaba ayuda para mejorar el \textbf{detector}, que es el
dispositivo que se encarga de extraer información de una corriente de radiofrecuencia modulada, y aunque ya él había desarrollado un
\textbf{detector magnético}, este solo brindaba una señal de frecuencia de audio a un receptor de teléfono. Un \textbf{detector}
confiable que pudiera guiar un instrumento de impresión era necesario. Fleming pudo desarrollar un \textbf{tubo al vacío} como resultado de su 
trabajo con \textbf{bombillas de efecto Edison}, a estas las denominó \textbf{válvulas de oscilación} ya que pasaba corriente en una sola
dirección. Fleming presentó una patente para estos tubos, cedida a la \emph{compañía Marconi} en el Reino Unido en noviembre de 1904 y esta
se emitió en septiembre de 1905. Conocida más tarde como la \textbf{válvula Fleming}, la \textbf{válvula de oscilación} se desarrolló con el
fin de rectificar la corriente de radiofrecuencia como componente detector de circuitos receptores de radio.\\
\indent En el propio siglo XIX ingenieros de telégrafos y teléfono habían reconocido la necesidad de incrementar la distancia que la señal pudiera ser
transmitida. En 1906 \emph{Robert Von Lieben} solicitó una patente para un \textbf{tubo de rayos catódicos} que usaba una bobina de deflexión
magnética externa y estaba destinado a usarse como amplificador en equipos de telefonía. A \emph{Lee de Forest} se le acredita la invención del
tubo triodo en 1907, el cual tenía la capacidad de amplificar las señales, y que fue el primero de su tipo que tuvo uso práctico. Sin embargo estos
\textbf{tubos de vacío} utilizados para amplificar la música y la voz que hicieron posibles las llamadas de larga distancia, creaban mucho calor y
se quemaban muy rápido, requiriendo alto mantenimiento. \textcite{wikipedia_2022_tube, wikipedia_2022_triode}\\
\indent La \textbf{ENIAC}(\emph{Electronic Numerical Integrator and Computer}) fue la primera computadora en usar los \textbf{tubos de vacío}, exactamente
18 000 de estos para poder funcionar, que hicieron que aquel dispositivo ocupara el tamaño de una habitación completa. Estos tubos permitían que las
señales fueran enviadas y los cálculos realizados de forma más rápida a través del uso de conmutación eléctrica en vez de conmutación mecánica. 
Debido al enorme consumo de energía eléctrica de la \textbf{ENIAC}, muchas personas creyeron que esta se destruiría, sin embargo los \textbf
{tubos de vacío} le permitieron soportar y funcionar. ~\textcite{richards_2022}.
