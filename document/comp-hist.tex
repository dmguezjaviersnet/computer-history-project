\documentclass[12pt,oneside]{uhthesis}
\usepackage{subfigure}
\usepackage[ruled,lined,linesnumbered,titlenumbered,algochapter,spanish,onelanguage]{algorithm2e}
\usepackage{amsmath}
\usepackage{amssymb}
\usepackage{amsbsy}
\usepackage{caption,booktabs}
\captionsetup{ justification = centering }
%\usepackage{mathpazo}
\usepackage{float}
\setlength{\marginparwidth}{2cm}
\usepackage{todonotes}
\usepackage{listings}
\usepackage{xcolor}
\usepackage{multicol}
\usepackage{graphicx}
\floatstyle{plaintop}
\restylefloat{table}
\addbibresource{Bibliography.bib}
% \setlength{\parskip}{\baselineskip}%
\renewcommand{\tablename}{Tabla}
\renewcommand{\listalgorithmcfname}{Índice de Algoritmos}
%\dontprintsemicolon
\SetAlgoNoEnd

\definecolor{codegreen}{rgb}{0,0.6,0}
\definecolor{codegray}{rgb}{0.5,0.5,0.5}
\definecolor{codepurple}{rgb}{0.58,0,0.82}
\definecolor{backcolour}{rgb}{0.95,0.95,0.92}

\lstdefinestyle{mystyle}{
    backgroundcolor=\color{backcolour},   
    commentstyle=\color{codegreen},
    keywordstyle=\color{purple},
    numberstyle=\tiny\color{codegray},
    stringstyle=\color{codepurple},
    basicstyle=\ttfamily\footnotesize,
    breakatwhitespace=false,         
    breaklines=true,                 
    captionpos=b,                    
    keepspaces=true,                 
    numbers=left,                    
    numbersep=5pt,                  
    showspaces=false,                
    showstringspaces=false,
    showtabs=false,                  
    tabsize=4
}

\lstset{style=mystyle}

\title{Transistor, microchip y microprocesador}
\author{\\\vspace{0.25cm}Javier E. Domínguez Hernández. C-412\\\vspace{0.2cm} David Orlando De Quesada Oliva. C-411}
% \advisor{\\\vspace{0.25cm}Yudivián Almeida\\\vspace{0.2cm}Carlos Bermudez Porto}
% \degree{Licenciado en (Matemática o Ciencia de la Computación)}
\faculty{Facultad de Matemática y Computación}
\date{Fecha\\\vspace{0.25cm}20 de octubre de 2022}
\logo{Graphics/uhlogo}
\makenomenclature

\renewcommand{\vec}[1]{\boldsymbol{#1}}
\newcommand{\diff}[1]{\ensuremath{\mathrm{d}#1}}
\newcommand{\me}[1]{\mathrm{e}^{#1}}
\newcommand{\pf}{\mathfrak{p}}
\newcommand{\qf}{\mathfrak{q}}
%\newcommand{\kf}{\mathfrak{k}}
\newcommand{\kt}{\mathtt{k}}
\newcommand{\mf}{\mathfrak{m}}
\newcommand{\hf}{\mathfrak{h}}
\newcommand{\fac}{\mathrm{fac}}
\newcommand{\maxx}[1]{\max\left\{ #1 \right\} }
\newcommand{\minn}[1]{\min\left\{ #1 \right\} }
\newcommand{\lldpcf}{1.25}
\newcommand{\nnorm}[1]{\left\lvert #1 \right\rvert }
\renewcommand{\lstlistingname}{Ejemplo de código}
\renewcommand{\lstlistlistingname}{Ejemplos de código}

\begin{document}

\frontmatter
\maketitle

% \include{FrontMatter/Dedication}
% \include{FrontMatter/Thanks}
% \include{FrontMatter/SupervisorOpinion}
\begin{resumen}
	Una síntesis de que personalidades participaron a lo largo de la historia en la
	creación del transistor, el microchip y el microprocesador, y como fue el camino
	para la invención de estos importantes componentes que forman parte no solo de la
	computadora actual, sino también de la gran mayoría de los dispositivos electrónicos
	que usamos en el día a día. 
\end{resumen}

\begin{abstract}
	A synthesis of which personalities participated throughout history in the
	creation of the transistor, the microchip and the microprocessor, and how was the path
	for the invention of these important components that are part not only of the
	current computer, but also the vast majority of electronic devices
	that we use on a daily basis.
\end{abstract}

\tableofcontents
\listoffigures
% \listoftables
% \listofalgorithms
% \lstlistoflistings

\mainmatter

% \include{MainMatter/Introduction}
\chapter{El transistor}\label{chapter:transistor}

\chapter{El microprocesador}\label{chapter:microprocesador}

\chapter{El microchip}\label{chapter:microchip}


\backmatter

% \include{BackMatter/Conclusions}
% \include{BackMatter/Recomendations}
\printbibliography[heading=bibintoc]




\end{document}
